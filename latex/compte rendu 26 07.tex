% !TEX TS-program = pdflatex
% !TEX encoding = UTF-8 Unicode

% This is a simple template for a LaTeX document using the "article" class.
% See "book", "report", "letter" for other types of document.

\documentclass[11pt]{article} % use larger type; default would be 10pt

\usepackage[french]{babel}
\usepackage[utf8]{inputenc} % set input encoding (not needed with XeLaTeX)

\usepackage{xcolor}
%%% Examples of Article customizations
% These packages are optional, depending whether you want the features they provide.
% See the LaTeX Companion or other references for full information.

%%% PAGE DIMENSIONS
\usepackage{geometry} % to change the page dimensions
\geometry{a4paper, top = 20mm} % or letterpaper (US) or a5paper or....
% \geometry{margin=2in} % for example, change the margins to 2 inches all round
% \geometry{landscape} % set up the page for landscape
%   read geometry.pdf for detailed page layout information

\usepackage{graphicx} % support the \includegraphics command and options

% \usepackage[parfill]{parskip} % Activate to begin paragraphs with an empty line rather than an indent

%%% PACKAGES
\usepackage{booktabs} % for much better looking tables
\usepackage{array} % for better arrays (eg matrices) in maths
\usepackage{paralist} % very flexible & customisable lists (eg. enumerate/itemize, etc.)
\usepackage{verbatim} % adds environment for commenting out blocks of text & for better verbatim
\usepackage{subfig} % make it possible to include more than one captioned figure/table in a single float
% These packages are all incorporated in the memoir class to one degree or another...

%%% HEADERS & FOOTERS
\usepackage{fancyhdr} % This should be set AFTER setting up the page geometry
\pagestyle{fancy} % options: empty , plain , fancy
\renewcommand{\headrulewidth}{0pt} % customise the layout...
\lhead{}\chead{}\rhead{}
\lfoot{}\cfoot{\thepage}\rfoot{}

%%% SECTION TITLE APPEARANCE
\usepackage{sectsty}
\allsectionsfont{\sffamily\mdseries\upshape} % (See the fntguide.pdf for font help)
% (This matches ConTeXt defaults)

%%% ToC (table of contents) APPEARANCE
\usepackage[nottoc,notlof,notlot]{tocbibind} % Put the bibliography in the ToC
\usepackage[titles,subfigure]{tocloft} % Alter the style of the Table of Contents
\renewcommand{\cftsecfont}{\rmfamily\mdseries\upshape}
\renewcommand{\cftsecpagefont}{\rmfamily\mdseries\upshape} % No bold!

%%% END Article customizations

%%% The "real" document content comes below...

\title{Compte rendu de réunion}
%\author{The Author}
%\date{} % Activate to display a given date or no date (if empty),
         % otherwise the current date is printed 

\begin{document}
\maketitle

\section*{Problème 1}
Le gradient de Lagrangian ne converge pas vers 0.

\subsection*{raison}

Le gradient n'est pas bien calculé.

J'ai utilisé \texttt{fddp.Vx} en tant que le dual de contraintes dynamique. Mais \texttt{fddp.Vx != kkt.dual} ici! La différence des 2 valeurs ne sont pas 0.

\section*{Problème 2}
Dans le modèle augmenté, j'ai seulement modifié la dérivée \texttt{Lx} sans toucher \texttt{Lxx}. Mais Lagrangian augmenté a un effet négligeable sur la hessianne:

Ce que j'ai ajouté dans la fonction cout \texttt{L}: $a := \lambda^\top r$

Ce que j'ai ajouté dans la dérivée \texttt{Lx}: $\frac{\partial{a}}{\partial{x}} := \lambda^\top J$

\medskip
Ce que je dois ajouter dans la hessianne \texttt{Lxx}: 
$\frac{\partial^2{a}}{\partial^2{x}} := \lambda^\top \frac{\partial{J}}{\partial{x}}$

Par apport au cas de Least-Square, la terme $\lambda^\top \textcolor{red}{\frac{\partial{J}}{\partial{x}}}$ n'est pas négligeable. On peut négliger $\textcolor{red}{r}^\top \frac{\partial{J}}{\partial{x}}$ car $r$ est petit

\subsection*{solution}
Il faut avoir la fonction qui calcule $\frac{\partial{J}}{\partial{x}}$.
\subsubsection*{méthode 1: la dérivée analytique}
 Nicolas m'as dit que c'est dans Pinocchio mais je n'ai pas trouvé. 
\subsubsection*{méthode 2: différence fini}
Pour la calculer par la différence fini, il faut que j'ai la fonction $J$. Pourtant je n'ai que la valeur de Jacobianne $J(x)$ dans la classe data. La fonction est dans Pinocchio et c'est calculé implicitement.

Où je peux les trouver dans le pinocchio?
\end{document}
